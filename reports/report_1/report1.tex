% !TEX TS-program = pdflatex
\documentclass[a4paper,12pt]{article}

\usepackage[noTeX]{mmap} % T1 fonts + mathematics (Unicode)
\usepackage[utf8]{inputenc}	% translates various standard and other input encodings into a ?LATEX internal language?
\usepackage[english,russian]{babel} % english and russian language
\usepackage{amssymb,amsmath,amsthm,amsfonts} % math symbols
\usepackage{color} % text color
\usepackage{wrapfig, graphicx} % figures
\usepackage{listings} % for source code
\usepackage{tcolorbox}
\usepackage{indentfirst} % indent at the beginning of the paragraph
\usepackage{hyperref} % links
\usepackage{caption,subcaption} % subfigures
\usepackage{tikz} % package for creating graphics programmatically
\usetikzlibrary{matrix,positioning,decorations.pathreplacing,calc} % used libraries for tikz package
\usepackage[margin=2cm]{geometry} % document margins
\usepackage{multirow}
\usepackage{stackengine}
\usepackage{xcolor}

\newcommand{\HRule}{\rule{\linewidth}{0.5mm}}
\newcommand{\quotes}[1]{``#1''}
%\numberwithin{equation}{subsection}
\newcommand{\noindex}{\hspace*{-0.8em}}
\DeclareMathOperator*{\argmax}{\arg\!\max}
\DeclareMathOperator*{\sign}{sign}
\DeclareMathOperator*{\rank}{rank}
\DeclareMathOperator{\Span}{span}
\newcommand{\matlabel}[2]{% \matlabel{<label>}{<stuff>}
  \begin{array}{@{}c@{}} \mbox{\small$#1$} \\ #2 \end{array}
}
\DeclareMathOperator{\Mcol}{col}
\DeclareMathOperator{\Mrow}{row}
\DeclareMathOperator{\Mnull}{null}

\def\mathrlap{\mathpalette\mathrlapinternal}
\def\mathclap{\mathpalette\mathclapinternal}
\def\mathllapinternal#1#2{\llap{$\mathsurround=0pt#1{#2}$}}
\def\mathrlapinternal#1#2{\rlap{$\mathsurround=0pt#1{#2}$}}

\newcommand{\lstfont}[1]{\color{#1}\scriptsize\ttfamily}

\definecolor{cugreen}{RGB}{17,185,0}

\usepackage{chngcntr}
\AtBeginDocument{\counterwithin{lstlisting}{subsection}}

\setcounter{secnumdepth}{0}

\begin{document}

\begin{center}
	\section{Отчет 1}
	(\today)
\end{center}

\newpage
\tableofcontents

\newpage
\subsection{Постановка задачи}
Исходная задача заключается в том, чтобы сообщить спутнику, вращающемуся по орбите Земле, такую начальную скорость, при которой спутник отдалится на значительное расстояние $R$ от Земли. Возможно следует искать начальную скорость, при которой спутник отдалится от Земли на расстояние $R < \infty$.

Следует отдельно отметить, что \quotes{основной} мусор, который летает на орбите Земли, находится в промежутке между орбитой МКС и орбитой захоронения: $300\text{км} - 1500\text{км}$ над поверхностью Земли или $6671\text{км} - 7871\text{км}$ от центра Земли.

Можно привести пример двух траекторий при разных начальных скоростях, но при одинаковом направлении начальных скоростей, при которых в одном случае спутник улетает на далекое расстояние (за пределы точки окрестности радиуса $L_2$), как показано на (Рис. \ref{fig:sim_r3bp_v0:sim_r3bp_v0_inside}), в то время, как при небольшом уменьшении начальной скорости (на $\Delta v_0 = 0.0088$) спутник уже не может вылететь из окрестности, как показано на (Рис. \ref{fig:sim_r3bp_v0:sim_r3bp_v0_outside}).

\begin{figure}[h!]
	\centering
	\begin{subfigure}{.495\textwidth}
		\includegraphics[width=0.94\textwidth]{../../img/sim_r3bp_v0_inside.png}
		\caption{Пример отсутствия \quotes{улета} спутника}
		\label{fig:sim_r3bp_v0:sim_r3bp_v0_inside}
	\end{subfigure}
	\begin{subfigure}{.495\textwidth}
		\includegraphics[width=0.94\textwidth]{../../img/sim_r3bp_v0_outside.png}
		\caption{Пример \quotes{улета} спутника}
		\label{fig:sim_r3bp_v0:sim_r3bp_v0_outside}
	\end{subfigure}
	\caption{Пример отсутствия \quotes{улета} спутника при уменьшении начальной скорости на небольшое значение}
\end{figure}

\begin{wrapfigure}{l}{0.45\textwidth}
	\vspace{-20pt}
	\begin{center}
    	\includegraphics[width=0.40\textwidth]{../../img/sim_r3bp_v0_in-outside.png}
	\end{center}
	\vspace{-20pt}
	\caption{Разница начальных скоростей}
	\label{fig:sim_r3bp_v0_in-outside}
    \vspace{10pt}
\end{wrapfigure}
На (Рис. \ref{fig:sim_r3bp_v0_in-outside}) приведены векторы обоих начальных скоростей. Приведенный график показывает, что незначительная разница между начальными скоростями приводит к совершенно разному поведению.

Однако мало учитывать тот факт вылетит ли спутник сквозь проход около точки $L_2$ или нет, надо также учитывать то, на какое расстояние и вреия требуется отдалить спутник. Можно сообщить такую начальную скорость, что спутник, после прохождения через окрестность точки $L_2$ будет двигаться практически прямолинейно от вращающихся Земли и Луны. Однако спутник при небольшой сообщенной начальной скорости не сможет улететь на бесконечность, потребуется достаточно много времени, чтобы он смог потратить всю свою кинетическую энергию, после чего начать возвращаться. Данный процесс можно видеть на (Рис. \ref{fig:oscillator}) приведены два графика: на первом виден процесс отдаления спутника от системы Земля-Луна, а на втором виден процесс возвращения.

\begin{figure}[h!]
	\centering
	\begin{subfigure}{.495\textwidth}
		\includegraphics[width=0.94\textwidth]{../../img/oscillator_1.png}
		\caption{Процесс отдаления спутника}
		\label{fig:oscillator_2}
	\end{subfigure}
	\begin{subfigure}{.495\textwidth}
		\includegraphics[width=0.94\textwidth]{../../img/oscillator_2.png}
		\caption{Процесс возвращения спутника}
		\label{fig:oscillator_1}
	\end{subfigure}
	\caption{Процесс отдаления и возвращения спутника}
	\label{fig:oscillator}
\end{figure}

% Привести пример двух траекторий, первая из которых показывает удаление на большое расстояние, а вторая -- тот факт, что спутник возвращается.
\subsection{Потенциал Роше}
Учитывая тот факт, что интеграл Якобы -- закон сохранения энергии, его можно переписать в виде: \[ v^2 - 2U = C\ \ \Leftrightarrow\ \ \frac{1}{2}v^2 - U = \frac{1}{2}C\ \ \Leftrightarrow\ \ T-\Pi=\widetilde{C}, \] где $T$ -- кинетическая энергия, которая зависит от квадрата скорости, $\Pi$ -- потенциальная энергия, которая зависит от потенциального поля. Потенциальное поле в круговой ограниченной задаче трех тел называется потенциалом Роше и в канонической системе единиц измерения вычисляется по формуле: \[ \Pi(x,y) = \frac{1}{2}(x^2+y^2) +\frac{1-\mu}{\rho_1} + \frac{mu}{\rho_2}, \] где $\rho_1$ -- расстояние от спутника до Земли, $\rho_2$ -- расстояние от спутника до Луны.
\begin{figure}[h!]
	\centering
	\includegraphics[width=\textwidth]{../../img/rochePotential_3.png}
	\caption{Потенциал Роше $\Pi(x,y)$}
	\label{fig:RochePotential}
\end{figure}

% Как получается скорость "ухода" на бесконечность (T - П = 0). Это очень затратно (рассчитать полученную скорость в системе единиц СИ для орбиты захоронения).

\end{document}


